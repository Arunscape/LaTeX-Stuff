
%preamble
\documentclass{article}
\synctex=1
\usepackage{graphicx}
\usepackage{hyperref}
%title page
\title{Blade Runner movie notes}

\author{Arun Woosaree}

%actual document
\begin{document}
\section{April 4}
Game technology participatory narrative

books vs games
read - play
character - avatars
plot - mission
setting - VIRTUAL WORLD
author - game studio, nerds, gamers


google search violence in video games
https://www.sciencedaily.com/releases/2018/01/180116131317.htm
University of York

The article says no evidence to supportlink between violent video games and behavior


ted tallk gaming can make a better world:
jane m

solve problems like hunger, climate change,
people need to play more games

she argues that video games are collaborative in nature,
and people right away trust you with a critical mission
the missions are matched to you ability
there's no sitting around
she doesn't talk about the addictive nature of videogames

\section{April 6}
People in favour of video games:
Brenda Romero ted talk - related to an exam question

Talks about a variety of games - Olympics (Hockey), baseball, Monopoly, AAA video games

She talks about how her daughter learned about the \textbf{middle passage} (black history slave trade),
and how it brought an emotional response in learning, while her daughter came home from school
not fully understanding the emotional impact of this significant historical event in American history

And other way that games can capture emotion in portraying different topics, relating to culture

\begin{itemize}
 \item middle passage
 \item Irish roots, irish potato famine
 \item the Holocaust
 \item trail of tears (forces relocation of native people)
 \item Mexican kitchen workers (illegal immigration)
\end{itemize}

In terms of numbers, choose one event and see the scope of people impacted by the event:
Trail of tears - hundreds of thousands of people relocated \url{https://en.wikipedia.org/wiki/Trail_of_Tears}

Holocaust - Millions of people


Brenda Romero believes that games can be used as a medium for understanding these historical events.


What if we were to make a game for a non University STEM student to understand how hard university is?
Design one on the spot. How do you earn points?

\begin{enumerate}
 \item Everything you do is given a rating from 0 to 4
 \item You're not allowed to sleep
\end{enumerate}

\begin{itemize}
 \item pokemon style - walk around do battles
 \item a wild midterm appears - fight or run away?
 \item you have 3 finals to do in 27 hours - go into MAXIMUM OVERDRIVE MODE
 \item collect items - coffee, caffiene pills work like health kits
 \item if you ae caught cheating, you get hit with the BAN HAMMER
 \item your sleep meter will always be low, but items like coffee and caffiene pills can bring up the sleep metre slightly
 \item each `week' you have certain objectives, and they are all timed. You get 7 or more objectives per week.
 \item by random chance you are assigned a TA who is either amazing, or complete garbage. Your score on certain objectives are affected by the TA,
       even though you might have put in more work than another player
 \item there are several non playable characters (NPC)s who you do not know the name of, but you frequently talk to them
 \item just like TAs, you get assigned different proffessors, who will help you on your journey. Some are more helpful than others.
       Your mark might not directly depend on the professors, but the help they give have varying levels of usefulness
 \item if your objective is part of the math department, your score is not relative to other players, which means
       everyone does poorly
 \item in other departments, a thing called the bell curve boosts your score relative to other players
 \item while you grind away, you look at your non STEM NPCs seem to be enjoying themselves
 \item also you have to make money but also pay the university to continue studying
\end{itemize}

Kind of make it a narrative RPG game where  you choose how the character develops by who the character interacts with and how.
\end{document}
