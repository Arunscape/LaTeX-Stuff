%preamble
\documentclass{article}

%sets length of spacing between paragraphs
\setlength{\parskip}{1em}
\usepackage{url}

\synctex=1

%temporary
\usepackage{blindtext}

%\usepackage[style=ieee]{biblatex}
%IEEE style bibliography
\bibliographystyle{IEEEtran}
%title page
\title{Why the Tesla Model 3 Should be Your Next Car\\
\vspace{.25cm}\large ENGL 199 Descriptive Report \vspace{-.5cm}}
\author{\LARGE Arun Woosaree}
\date{\today}
%actual document
\begin{document}
  \maketitle %insert titlepage here

  %Introduction
  \section{Introduction}
  %General intro statement
  %Talking about Autopilot
  %Talking about price
  One might think that a Tesla vehicle would be expensive, given the many
  features a Tesla car has to offer. While the recently announced Tesla Roadster
  has a base price close to a quarter million\cite{teslaroadster}, one may be
  surprised with Tesla's other consumer offerings, specifically the Model 3.
  I recommend the Model 3 for the average person,
  given its more budget-friendly price point, with all the characteristic features
  one would expect from a Tesla vehicle.
  %Talking about the Environment
  \newpage

  \section{Autopilot}
  Autopilot is one of the first amazing features you may notice about Tesla vehicles,
  offering both convenience and safety. Currently, all vehicles in production by Tesla
  have sufficient hardware and software to drive without any human interaction.
  A Tesla car can be summoned with the click of a button, predict where you're
  going without you telling it, know that it should stop en route to charge if needed,
  and park itself without any user input. Tesla even claims that their cars are capable of driving
  at a safety level ``substantially greater'' than a human driver\cite{allcarsautopilot}.
  Tesla is able to accomplish this by including eight cameras, twelve ultrasonic sensors, and a front radar
  sensor which all work together to monitor the area around the car. It is not hard
  to see why Tesla makes such bold claims about how effective its Autopilot system is,
  because a mere human is physically incapable of constantly keeping track of the area around
  the car he is driving. A human needs to divide his attention between looking forward,
  checking the blind spot when changing lanes or parking, among other things one would typically
  look out for when driving. In contrast, a computer is able to monitor everything
  around the car simultaneously. The ultrasonic and radar sensors
  provide information to the computer for sensing objects, that a human cannot
  detect with his natural senses. For example, the radar sensor is very useful when
  there is thick fog that a camera or human cannot see through. The computer, powered
  by the powerful NVIDIA DRIVE PX 2 chip uses machine learning to improve driving for
  thousands of users every day. Other companies like GM have to test their autonomous
  driving systems with fleets monitored by their own employees, while Tesla collects
  real-world data (with consent) from its users and processes the data with machine
  learning to constantly push new updates and features to improve the autonomous
  driving experience.
  \cite{tdata1}\cite{tdata2}\cite{tdata3}
  The amount of real-world data Tesla has, combined with its sensors (which can
  detect danger faster than a human can react\cite{predictcrash}) makes Tesla's
  autonomous driving system the most reliable on the market for consumers today.

  %Reason 2
  \section{Price}
  Price is a concern for most who are in the market for a new vehicle. Upon seeing
  all the features a Tesla car has to offer, one might think that surely,
  such a futuristic vehicle would cost a pretty penny. While Tesla does have
  more luxurious offerings like the Roadster and the Model S, I would argue that the
  Tesla Model 3 is ideal for most users, starting at \$35 000 USD.\@ This advertised
  price might seem deceptive until you consider that it is common practice
  in the auto industry to advertise a starting price, while most of the models available
  are sold at a premium of a few thousand dollars more expensive than the base model. The
  Tesla Model 3 is no exception here, with a maxed out model costing about
  \$59 000 USD\cite{maxprice}. However, if you don't care about `premium upgrades',
  19-inch wheels, or a custom paint colour, then the price will be closer to
  \$50 000 USD, if you opt to have `Full Self-Driving Capability', and the larger
  battery. It should be noted that Full Self-Driving Capability does not need to be
  bought at the time of purchase, and can be unlocked later if you so choose.
  \$50 000 might seem expensive to some for a car, but I would argue that the price is well worth
  it for all of the features you get that are not offered in similarly priced cars
  Additionally, depending on where you live, there is a good chance that there are
  tax benefits with purchasing an electric vehicle, not to mention the long-term savings.
  Using electricity to power a Tesla vehicle is much cheaper than using gasoline in
  a modern car. Taking the US average of twelve cents per kilowatt hour, it would
  cost about nine dollars to completely charge a Tesla Model 3 with the `Long Range'
  ugrade. On one charge, the Model 3 with the long range upgrade has a range of about
  500 km\cite{500km}. Compared to a somewhat best-case scenario for a gasoline powered vehicle,
  if the cost of gasoline is \$0.75 per litre\cite{75cents}, and a car consumes a conservative five
  litres of gasoline per 100 km\cite{priusmpg}, it would cost a little over \$18 dollars, which is
  twice the cost of charging the Tesla Model 3 with the long range upgrade. Elon Musk claims that
  ``If you fill up a sports car, you'd spend 10 times as much.''\cite{rosoff_elon_nodate}. Another cost
  saving perk of Tesla vehicles, is that since they don't use gas engines, maintenance costs
  can be lower, since there is no need for regular oil changes, for example.\cite{oilchange}
  Since a car is a tool one would purchase for the long term, it is easy to
  recommend the Tesla Model 3, given the savings when you look at its long-term costs.
  %Long term savings
  % \cite{money}
  % \cite{cleanerevenwithcoal}

  %Reason 3
  \section{The Environment}
  \blindtext{}\blindtext{}

  %Conclusion
  \section{Conclusion}
  \blindtext{}\blindtext{}

\bibliography{references}
\end{document}
