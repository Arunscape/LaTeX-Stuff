% Letter class
\documentclass[letterpaper]{letter}

% Name of sender
\name{Arun Woosaree}

% Signature of sender
\signature{Arun Woosaree}

% Address of sender
\address
{
Arun Woosaree\\
Donadeo Innovation Centre
for Engineering\\
9211-116 Street NW\\
Edmonton, Alberta\\
Canada T6G 1H9\\
}

%-----------------------------------------------------------------------------%
\longindentation=0pt
\usepackage{block}
\usepackage{lipsum}
\usepackage{setspace}
\usepackage[colorlinks,urlcolor=blue]{hyperref}
\def\UrlBreaks{\do\/\do-}
\begin{document}

% Name and address of receiver
\begin{letter}
{
Wayne Defehr\\
Department of English and Film Studies\\
3-5 Humanities Centre\\
University of Alberta\\
Edmonton, Alberta\\
Canada T6G 2E5\\
}

% Opening statement
\opening{Dear Mr. Defehr,}
\doublespacing
% Letter body

Moore's Law is coming to an end due to the physical size of the silicon atom,
which will limit the size of transistors in semiconductor chips. Without the
ability to make smaller, more efficient transistors, computers cannot be made
faster while maintaining the same form factors as we know today. Unless research
is done to find other suitable semiconductors for making smaller transistors ,
we will hit a wall in computing where we are limited by processing speed of our
computers. Right now, carbon nanotubes may seem as a promising material for
building semiconductor chips, given that a team of researchers at the University
of Wisconsin-Madison recently built a carbon nanotube transistor which
outperformed silicon. Currently, however, carbon nanotubes are not feasable for
use in integrated circuits. For the most part, carbon nanotubes grown from
solids other than carbon (especially semiconductors) are too fragile for
practical use, and the gate voltages are rather high, which results in higher
power consumption. On the other hand, gallium nitride (GaN), which is already
known for its fantastic semiconductor properties shows a lot of promise, given
that it is already used in many industrial and military applications.
Datacenters are reducing their power consumption today by powering their
servers using power supplies with GaN to convert voltages more efficiently,
and Raytheon (a defense contractor company) makes radar systems which use GaN to
amplify radar radio frequencies. In addition to being smaller and more power
efficient, GaN chips are also much faster compared to their silicon-based
counterparts. Autonomous vehicles use LiDAR technology to fire beams of light in
order to map the area around themselves. Underlying this technology are GaN
chips, because GaN allows the laser beams to be fired at faster speeds than a
silicon chip could process. This allows for higher resolution 3-D maps to be
made, which has improved the quality and reliability of autonomous vehicles.
Given that GaN is already useful in all these applications today, a
microprocessor chip made from GaN would revolutionize the computing industry,
bringing new speed improvements to our daily computing needs. The main reason
we have not yet moved on from using silicon processors in computers today is
because of the economies of scale of silicon chips. It is very cheap to produce
silicon chips, since obtaining silicon is easy, and companies like
GlobalFoundries and Intel have been producing and improving the manufacturing
process of these chips for over 50 years. GaN is still relatively new,
which means that improvements definitely can be made with regards to finding a
cost-effective manufacturing process that would enable the mass production of GaN
based processor chips.


Yuji Zhao, an expert in electrical and computer engineering from Arizona State
University %who aims to build the first microprocessor from gallium nitride
notes that GaN chip design and fabrication requires special growth techniques
which are different compared to how silicon chips are manufactured today. Many
other researchers have already investigated different techniques of growing pure
GaN crystals. The crystals grown need to be pure, just like for silicon chips
since we are dealing with transistors smaller than the 12 nm silicon ones being
manufactured today. Since many methods are already known for growing pure GaN
crystals, my research would be concerned with improving these methods and
finding a cost-effective way to mass produce GaN crystals. The current practice
for growing silicon crystals, which is vapour deposition does not show much
promise for growing GaN crystals. However, growing crystals under high
temperature and pressure seems to show promise for GaN, since most leading
gallium nitride substrate manufacturers are doing this now. Ammono, a company in
Warsaw, regarded for its high quality GaN crystals is currently able to grow GaN
crystals measuring about 51 mm in diameter. Similar to how quartz crystals can
be grown in a high pressure chamber with supercritical water, GaN crystals can
be grown using high pressure and heat with ammonia. This process is known as \textit{amonnothermal growth} Without a doubt, this
process can be improved upon to get higher yields. Ideally  the process of
growing these GaN crystals could be improved to a point where GaN crystals can
be grown about as large as silicon crystals are in semiconductor manufacturing, which
typically have a diameter of about 300mm. Another advantage is that if we can get a GaN crystal to
grow as large as we can grow silicon crystals today, each wafer can be
sliced thinner, which results in higher yields. GaN crystals can be sliced thinner than silicon,
because it is stronger than silicon. One method that could be used to improve the growth of the
GaN crystals is to  to dope the crystals, similar to how silicon manufacturers dope
their crystals with small amounts of boron, aluminium, and various other elements.
By trying out different doping agents, and refining the ammonothermal process,
progress will be made towards being able to mass produce GaN crystals in a more
cost-effective fasion than they are produced today.



As a second year Computer Engineering Co-op student at the University of
Alberta, it would be exciting to contribute research that would make gallium
nitride based processors a reality. I believe my background in computer
engineering and tendency to always be on the bleeding edge of technology make me
well-suited for this type of research. So far during my degree, I have been
exposed to digital logic, time signals, assembly language, C programming, and
chemistry which are all essential for this type of research which involves
working on the foundations of next-gen computing devices.

% Closing statement
\closing{Sincerely,}

\end{letter}
\newpage
\textbf{Note:}

\textbf{Links visited:}
\begin{itemize}
  \item \url{http://fortune.com/2016/06/11/raytheon-next-gen-chips/}
  \item \url{https://arstechnica.com/information-technology/2016/06/cheaper-better-faster-stronger-ars-meets-the-latest-military-bred-chip/}
  \item \url{https://www.allaboutcircuits.com/news/how-carbon-nanotubes-could-help-replace-silicon-in-chip-fabrication/}
  \item \url{https://www.berkeley.edu/news/media/releases/2003/04/09_tubes.shtml}
  \item \url{https://pdfs.semanticscholar.org/3fb6/afb5918951c44db170da745cc6aeb326da10.pdf}
  \item \url{http://www2.lbl.gov/Science-Articles/Archive/MSD-gallium-nitride-nanotube.html}
  \item \url{http://www.nbi.dk/~nygard/Integration_of_Carbon_Nanotubes_Stobbe_et_al.pdf}
  \item \url{https://www.sciencedaily.com/releases/2014/08/140827122509.htm}
  \item \url{https://www.cnet.com/news/life-after-silicon-how-the-chip-industry-will-find-a-new-future/}
  \item \url{http://epc-co.com/epc}
  \item \url{https://spectrum.ieee.org/nanoclast/semiconductors/materials/carbon-nanotube-transistors-finally-outperform-silicon}
  \item \url{https://news.wisc.edu/for-first-time-carbon-nanotube-transistors-outperform-silicon/}
  \item \url{https://www.raytheon.com/news/feature/power_patriot}
  \item \url{https://phys.org/news/2017-12-gallium-nitride-processornext-generation-technology-space.html}
  \item \url{https://www.allaboutcircuits.com/news/gan-gaining-traction-one-chip-at-a-time/}
  \item \url{https://venturebeat.com/2015/04/02/move-over-silicon-gallium-nitride-chips-are-taking-over/}
  \item \url{https://na.industrial.panasonic.com/products/semiconductors/microcontrollers}
  \item \url{http://www.beck-shop.de/fachbuch/leseprobe/9783642048289_Excerpt_001.pdf}
  \item \url{https://spectrum.ieee.org/semiconductors/materials/the-worlds-best-gallium-nitride}
  \item \url{http://www.dtic.mil/dtic/tr/fulltext/u2/a444058.pdf}
\end{itemize}
\vspace*{\fill}
This is an assignment for an English class. Therefore, it is not a real research
proposal letter, and I cannot guarantee the validity of the information
presented due to potential misinterpretations of mine. Also, it should be
obvious but, how would it be possible for a second year student to do something
as advanced as building a new computer processor lol.
\end{document}
