\documentclass[letterpaper,12pt]{article}
\synctex=1

\usepackage[margin=1in]{geometry}
\usepackage{lipsum}

\usepackage{setspace}
\doublespacing
\usepackage{parskip}
\parskip=\baselineskip

\usepackage[hyphens,spaces,obeyspaces]{url}
\usepackage{hyperref}

\usepackage{graphicx}

\interfootnotelinepenalty=10000


% noto sans font
% \usepackage[sfdefault]{noto}
% \usepackage[T1]{fontenc}

% \usepackage[
%   style = IEEEtran,
% ]{biblatex}
\usepackage[
    style=ieee,
    backend=biber
    ]{biblatex}
\addbibresource{references.bib}

\title{WKEXP 902 Work Term Report}
\author{Arun Woosaree \\ \\ XXXXXXX}


\begin{document}
\relax
\begin{titlepage}
 \maketitle
 \thispagestyle{empty} %because I wanted to use \maketitle
 \centering
 \large
 \vspace{1cm}
 Work term 2\\
 \vspace{1cm}
 Computer Engineering Software Co-op \\
 \vspace{1cm}
 Coordinator: Linda Szekely
\end{titlepage}

\section{Introduction}
The purpose of this report is to develop a profile of an industry which is of interest to me.
By exploring different aspects of the industry of Artificial Intelligence, or AI.
It can be noted that ``Machine Learning'' is a subfield of AI \cite{sasAIvsML} , such as its history, 
We wil also cover the projected economic status 

\section{Brief History of the Industry}
Contrary to what one might think, the idea of a machine being able to reason is nothing new. 
Early Philosophers, such as René Descartes ($\sim$1600s C.E.) have used the idea of a "mechanical man"
to define what it is to be human.\cite{briefhistory}\cite{sep-descartes}
Another philosopher, named Etienne Bonnot, de Condillac\cite{briefhistory} ($\sim$1700s C.E.)
thought about "an originally inanimate and insentient human being"\cite{sep-condillac}
and how much information it would need to acquire by exposing it to different sensations
before it would become intelligent. However, an example more familiar to us nowadays
might be this quote from the \textit{Wizard of Oz}\cite{wizardofoz}:
\begin{quotation}
    \noindent“Scarecrow: I haven't got a brain... only straw.\\
    Dorothy: How can you talk if you haven't got a brain?\\
    Scarecrow: I don't know... But some people without brains do an awful lot of talking... don't they?\\
\end{quotation}

Before the Wizard of Oz, however, people were already attempting to make physical machines
appear intelligent. So-called `automatons', or self-operating machines have existed for
a very long time. One of the first humanoid automatons ($\sim$800 C.E.) actually played the flute, and was programmable.\cite{automaton}
Other automatons have been created to write letters, draw art, and a lot more.

Automatons though, are not intelligent.
They must be programmed to do a pre-defined task, and their actions were
usually defined by a clock system.
Today, the game of chess is sometimes used as a measure of intelligence.
In the 1700s, an ``automaton'' called ``The Turk'' was made, which appeared to play chess
against human players autonomously\cite{turk}. The Turk was not a true automaton, since
it was actually a hoax, controlled by a human. In 1912, however, the ``El Ajedrecista''
was made by Leonardo Torres \cite{actualchessmachine}. It was not made to play chess from the
very beginning with a human, rather, it implemented an algorithim for a
specific end-game scenario. That is, the scenario when the human player has
only a King, and the automaton has only a Rook and a King. Although it did not
checkmate the human player in the minimum number of moves possible, it would
eventually do it.
This automaton is considered by some, the first computer game ever.
It was not until the 1960s that a computer was able to fully play chess against a human player.\cite{computerchess}
On May 11, 1997, the man considered to be the "world's best chess player" officially lost to Deep Blue,
a chess AI owned by IBM.\cite{computerchess}


The notion that an artificial creation can be made smarter than humans is a scary one to some,
and with no doubt, there exists a plethora of dystopian stories about machines being a threat to humans.
A popular example is Blade Runner, where the artificial creations rival humans
at almost everything, except for feeling and displaying emotions. (However, there is one so-called `replicant'
who does feel emotions)\cite{bladerunner}. Another famous example is from HAL 9000 in \textit{A Space Odyssey}
where the sentient AI detects that the humans want to shut it off, and says the
famous quote: \cite{HAL} 
\begin{quote}
   I know that you and Frank were planning to disconnect me. And I'm afraid that's something I cannot allow to happen.
\end{quote}

From these fictional stories, however, comes the inspiration for AI-powered things we interact with
today, such as virtual assistants (Google, Siri, Alexa\dots) 



\section{Recent Technological Advances}

In the ``brief history'' section above, a common theme was making machines
which appear human. A modern example today, is Sophia, a humanoid robot
who is the first one to recieve citizenship from any country.
Although not a true general intelligence\cite{sopiaAI}, Sophia uses various 
technologies like Google's natural language APIs.

Although beating the world's chess champion was considered a major
victory for the advancement of AI,
a game which is much more complex is Go.
Go is a game which originates from Ancient China,
which has relatively simple rules, yet the amount of legal
moves each turn is vastly larger. While chess is estimated to have about $10^{120}$ 
possible games, 
In 2017, Google Deepmind's AlphaGo program was the first program which beat the world's
best Go player. Not only is this facinating in and of itself, but the 
way the program worked was fundamentally different compared to previous 
programs which played Go, and even chess. Normally, these programs
use algorithms and evaluate different outcomes of future moves
in a `tree'. However, Alphago used a neural network which was trained
using machine learning.\cite{alphagopaper}. The program was never told how to
actually play the game, nor did it learn by watching a human play
\footnote{Actually, AlphaGo initially trained on datasets with humans playing,
and some heuristics were hand crafted. However, AlphaGo Zero, and AlhpaZero
were made afterwards based on no human data, and are significantly
better than their predecessors.}. It still
evaluates future moves, but the evaluation of the trees are `learned', and 
not pre-defined by a human.

It is widely known that Google uses AI to synthesize and recognize human speech,
among many other things. One impressive thing that was
achieved only recently was the
ability for Google's Assistant to sustain phone conversations with real
people, by booking an appointment on someone's behalf for example. \cite{googleduplex} 
There are other examples where we can see AI being incorporated in bleeding edge consumer products.
For example, NVIDIA uses DLSS (Deep Learning Super Sampling) in their latest consumer hardware to
upscale the resolution of real-time rendered 3-D graphics.\cite{dlss}
Though it isn't the first company to do so, Tesla uses AI to in their self-driving cars.\cite{teslaautopilot}
Amazon uses AI to suggest new products to you.\cite{aznai}

AI is being used in more than just consumer products though. Phillips,
an international electronics company is actively working on using
AI to improve their healthcare products in hospitals, and also 
solutions for hospitals to improve efficiency.


% \section{Economic Factors Influencing the Industry}
% https://www.newsmax.com/finance/richardagu/factors-artificial-intelligence-job/2018/04/24/id/856267/

% pretty much just demand for the technology and whether or not
% certain companies prefer the human touch over machines\dots

\section{Geographic locations of industry concentration}
% hey Edmonton actually has some pretty cool machine learning research stuff
% going on

In Canada, the locations with the highest concentration of AI-focused
entities are in Toronto, Edmonton, and Montreal.\cite{investincanada}

Edmonton is home to amii, or the Alberta Machine Intelligence Institute.
Additionally, ``The University of Alberta, located in Edmonton, is ranked \#2 in the world for AI and ML research.''\cite{investincanada}\cite{edmonton.ai}.
At the U of A, several companies are partnered for the purposes of AI research.
There's Google Deepmind, IBM, ATB alphabeta, and RBC Borealis to name a few.

Internationally, there's obviously the San Francisco Bay Area,
but other hubs for AI include
the Boston-New York area,
London,
Beijing,
Shenzen,
Berlin.
and Bangalore to name a few.\cite{aihubs}\cite{aihubsreview}

\section{Major employers}

fuck don't forget to cite this lol https://thenextweb.com/artificial-intelligence/2018/07/05/companies-work-ai-technology/

The obvious major employers that first come to mind when one thinks of AI are:
Google,
Tesla,
IBM,
NVIDIA, Intel, AMD,
Facebook,
Microsoft,
Phillips (for healthcare technology),
Panasonic (computer vision),


Other major companies which don't immediately come to mind, but totally make sense when you think about it are:
GM,
Volkswagen (really, any major auto manufacturer),
Uber,
Netflix,
Spotify,

Some may be surprised to know that , , and even banks are also investing in AI technologies.
JPMorgan Chase for example, and actually, here in Edmonton, there's ATB Financial and RBC

A thing to be wary of, is a lot of startups and other companies claim to be ``AI-based''
but in reality, they don't use any AI at all. \cite{}

% \section{Projected Economic Status}
% todo \cite{forbesprojected}
% https://www.forbes.com/sites/louiscolumbus/2018/02/18/roundup-of-machine-learning-forecasts-and-market-estimates-2018/

\section{Government Involvement/Control}
If anything, AI has a lot of support from governments in general.
In 2017, the Canadian federal government included \$125 million in the
federal budget for the purposes of furthering Canada's role in the
advancement of AI technologies.\cite{canadafederalfunding}. 
Additionally, obviously little is known about what the military
branches of government are working on with AI, but it will obviously
be used as a tool for several things\footnote{this is purely speculation on my part},
such as aiding with 
strategization, improving weapon accuracy, and perhaps also improving
existing equipment.

One thing, howver, that is currently very strictly regulated
in most areas is self-driving vehicles. It's not surprising,
though that something which can impact public safety so much
would be regulated. Since we still have to decide as a society
who would be responsible in the case of a collision
involving a vehicle without a driver, most
jurisdictions still require that a human is behind the wheel on public roads.
One interesting exception is in Ontario, where a test involving a driverless car
requires either a human to be in the passenger seat, or that the car is being
remotelyh monitored. Either way, the vehicle must have at least \$5 million
in insurance coverage.\cite{ontariodriverless}

\section{Current hiring trends and long term prospects}

AI absolutely is a fad word as of the time of writing. However, given that continuous improvements
have been made on it since the 1900s, and it is still relevant, in my opinion already says something.
There will always be some companies who prefer, or even require the human touch
over a machine, however, I believe that AI will pervade our lives even more in the future.
In fact, the 
``International Data Corporation forecasts that spending on AI and machine learning will grow from \$12B in 2017 to \$57.6B by 2021.''.
\cite{forbesprojected}
Even though AI already pervades our lives today, 
it is still a field of active research\cite{uofaAI}, meaning that new
discoveries are yet to be made and applied in the real world.
Humans have certain limits, and so do computers. But, we can overcome many challenges by leveraging this technology,
even those that we haven't thought of yet. 
As mentioned earlier,
AI is being used to advance progress in a vast majority of fields
such as manufacturing automation, medicine, transport, and more.

\section{Conclusion}

AI has always been a field of interest for me.
It's always interesting to see the things that people are creating
using the technology. I actually plan to take relevant courses (ECE 449 and CMPUT 466)
to understand the math and logic that goes on behind the scenes
which makes the technology work.


\singlespacing
\nocite{*}
\printbibliography
% \bibliographystyle{IEEEtran}
% \bibliography{references}
\end{document}
