\documentclass[letterpaper,12pt]{article}
\synctex=1

\usepackage[margin=1in]{geometry}
\usepackage{lipsum}

\usepackage{setspace}
\doublespacing
\usepackage{parskip}
\parskip=\baselineskip

\usepackage[hyphens,spaces,obeyspaces]{url}
\usepackage{hyperref}

\usepackage{graphicx}


% noto sans font
% \usepackage[sfdefault]{noto}
% \usepackage[T1]{fontenc}

% \usepackage[
%   style = IEEEtran,
% ]{biblatex}
\usepackage[
    style=ieee,
    backend=biber
    ]{biblatex}
\addbibresource{references.bib}

\title{WKEXP 902 Work Term Report}
\author{Arun Woosaree \\ \\ XXXXXXX}


\begin{document}
\relax
\begin{titlepage}
 \maketitle
 \thispagestyle{empty} %because I wanted to use \maketitle
 \centering
 \large
 \vspace{1cm}
 Work term 2\\
 \vspace{1cm}
 Computer Engineering Software Co-op \\
 \vspace{1cm}
 Coordinator: Linda Szekely
\end{titlepage}

\section{Introduction}
The purpose of this report is to develop a profile of an industry which is of interest to me.
By exploring different aspects of the industry of Artificial Intelligence, or AI.
It can be noted that ``Machine Learning'' is a subfield of AI \cite{sasAIvsML} , such as its history, 
We wil also cover the projected economic status 

\section{Brief History of the Industry}
Contrary to what one might think, the idea of a machine being able to reason is nothing new. 
Early Philosophers, such as René Descartes ($\sim$1600s C.E.) have used the idea of a "mechanical man"
to define what it is to be human.\cite{briefhistory}\cite{sep-descartes}
Another philosopher, named Etienne Bonnot, de Condillac\cite{briefhistory} ($\sim$1700s C.E.)
thought about "an originally inanimate and insentient human being"\cite{sep-condillac}
and how much information it would need to acquire by exposing it to different sensations
before it would become intelligent. However, an example more familiar to us nowadays
might be this quote from the Wizard of Oz\cite{wizardofoz}:
\begin{quotation}
    \noindent“Scarecrow: I haven't got a brain... only straw.\\
    Dorothy: How can you talk if you haven't got a brain?\\
    Scarecrow: I don't know... But some people without brains do an awful lot of talking... don't they?\\
\end{quotation}

Before the Wizard of Oz, however, people were already attempting to make physical machines
appear intelligent. So-called `automatons', or self-operating machines have existed for
a very long time. One of the first humanoid automatons ($\sim$800 C.E.) actually played the flute, and was programmable.\cite{automaton}
Other automatons have been created to write letters, draw art, and a lot more.

automatons though, are not intelligent.
They must be programmed to do a pre-defined task.
Today, the game of chess is sometimes used as a measure of intelligence.
In the 1700s, an ``automaton'' called ``The Turk'' was made, which appeared to play chess
against human players autonomously\cite{turk}. The Turk was not a true automaton, since
it was actually a hoax, controlled by a human. In 1912, however, the ``El Ajedrecista''
was made by Leonardo Torres \cite{actualchessmachine}. It was not made to play chess from the
very beginning with a human, rather, it implemented an algorithim for a
specific end-game scenario. That is, the scenario when the human player has
only a King, and the automaton has only a Rook and a King. Although it did not
checkmate the human player in the minimum number of moves possible, it would
eventually do it.
This automaton is considered by some, the first computer game ever.
It was not until the 1960s that a computer was able to fully play chess against a human player.\cite{computerchess}
On May 11, 1997, the man considered to be the "world's best chess player" officially lost to Deep Blue,
a chess AI owned by IBM.\cite{computerchess}


The notion that an artificial creation can be made smarter than humans is a scary one to some,
and with no doubt, there exists a plethora of dystopian stories about machines being a threat to humans.
A popular example is Blade Runner, where the artificial creations rival humans
at almost everything, except for feeling and displaying emotions. (However, there is one so-called `replicant'
who does feel emotions)\cite{bladerunner}

ooh maybe you can talk about sophia AI here

From these fictional stories, however, comes the inspiration for AI-powered things we interact with
today, such as virtual assistants (Google, Siri, Alexa\dots) 

HAL 9000
I'm sorry Dave I'm afraid I can't do that


\section{Recent Technological Advances}
maybe talk about nvidia here

NVIDIA uses DLSS (Deep Learning Super Sampling) in their consumer hardware to
upscale the resolution of real-time rendered 3-D graphics.\cite{dlss}

It is widely known that Google uses AI to synthesize and recognize human speech.

Though it isn't the first company to do so, Tesla uses AI to in their self-driving cars.\cite{teslaautopilot}

Amazon uses AI to suggest new products to you.\cite{aznai}

Google deepmind

\section{Economic Factors Influencing the Industry}
https://www.newsmax.com/finance/richardagu/factors-artificial-intelligence-job/2018/04/24/id/856267/

pretty much just demand for the technology and whether or not
certain companies prefer the human touch over machines\dots

\section{Geographic locations of industry concentration}
hey Edmonton actually has some pretty cool machine learning research stuff
going on

Fun fact, ``The University of Alberta, located in Edmonton, is ranked \#2 in the world for AI and ML research.''
obviously there's Silicon Valley as well...

ATB alpha beta
also partnered with the U of A

The
Alberta
Machine
Intelligence
Institute
(Amii)

https://medium.com/syncedreview/2017-in-review-10-leading-ai-hubs-e6f4d8a247ee
https://www.cbc.ca/news/canada/edmonton/artificial-intelligence-deepmind-edmonton-google-research-1.4195026

\section{Major employers}

fuck don't forget to cite this lol https://thenextweb.com/artificial-intelligence/2018/07/05/companies-work-ai-technology/

The obvious major employers that first come to mind when one thinks of AI are:
Google,
Tesla,
IBM,
NVIDIA, Intel, AMD
Microsoft,
Phillips (for healthcare technology),
Panasonic (computer vision),


Other major companies which don't immediately come to mind, but totally make sense when you think about it are:
GM,
Volkswagen (really, any major auto manufacturer)

Some may be surprised to know that , , and even banks are also investing in AI technologies.
JPMorgan Chase for example, and actually, here in Edmonton, there's ATB Financial and RBC

A thing to be wary of, is a lot of startups and other companies claim to be ``AI-based''
but in reality, they don't use any AI at all. \cite{}

\section{Projected Economic Status}
todo \cite{forbesprojected}
https://www.forbes.com/sites/louiscolumbus/2018/02/18/roundup-of-machine-learning-forecasts-and-market-estimates-2018/

\section{Current hiring trends and long term prospects}
oof


\section{Conclusion}

AI absolutely is a fad word as of the time of writing. However, given that continuous improvements
have been made on it since the 1900s, and it is still relevant, in my opinion already says something.
Humans have certain limits, and so do computers. But, we can overcome many challenges by leveraging this technology,
even those that we haven't thought of yet. Even though AI already pervades our lives today,
it is still a field of active research\cite{uofaAI}, meaning that new
discoveries are yet to be made and applied in the real world.


\singlespacing
\nocite{*}
\printbibliography
% \bibliographystyle{IEEEtran}
% \bibliography{references}
\end{document}
