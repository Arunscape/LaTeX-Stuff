\documentclass[letterpaper,12pt]{article}
\synctex=1

\usepackage[margin=1in]{geometry}
\usepackage{lipsum}

\usepackage{setspace}
\doublespacing
\usepackage{parskip}
\parskip=\baselineskip

\usepackage[hyphens,spaces,obeyspaces]{url}
\usepackage{hyperref}

\usepackage{graphicx}


% noto sans font
% \usepackage[sfdefault]{noto}
% \usepackage[T1]{fontenc}

% \usepackage[
%   style = IEEEtran,
% ]{biblatex}
\usepackage[
    style=ieee,
    backend=biber
    ]{biblatex}
\addbibresource{references.bib}

\title{WKEXP 902 Work Term Report}
\author{Arun Woosaree \\ \\ XXXXXXX}


\begin{document}
\relax
\begin{titlepage}
 \maketitle
 \thispagestyle{empty} %because I wanted to use \maketitle
 \centering
 \large
 \vspace{1cm}
 Work term 2\\
 \vspace{1cm}
 Computer Engineering Software Co-op \\
 \vspace{1cm}
 Coordinator: Linda Szekely
\end{titlepage}

\section{Introduction}
The purpose of this report is to develop a profile of an industry which is of interest to me.
By exploring different aspects of the industry of Artificial Intelligence, or AI.
It can be noted that ``Machine Learning'' is a subfield of AI \cite{sasAIvsML} , such as its history, 
We wil also cover the projected economic status 

\section{Brief History of the Industry}
Contrary to what one might think, the idea of a machine being able to reason is nothing new. 
Early Philosophers, such as René Descartes ($\sim$1600s C.E.) have used the idea of a "mechanical man"
to define what it is to be human.\cite{briefhistory}\cite{sep-descartes}
Another philosopher, named Etienne Bonnot, de Condillac\cite{briefhistory} ($\sim$1700s C.E.)
thought about "an originally inanimate and insentient human being"\cite{sep-condillac}
and how much information it would need to acquire by exposing it to different sensations
before it would become intelligent. However, an example more familiar to us nowadays
might be this quote from the Wizard of Oz\cite{wizardofoz}:
\begin{quotation}
    \noindent“Scarecrow: I haven't got a brain... only straw.\\
    Dorothy: How can you talk if you haven't got a brain?\\
    Scarecrow: I don't know... But some people without brains do an awful lot of talking... don't they?\\
\end{quotation}

Before the Wizard of Oz, however, people were already attempting to make physical machines
appear intelligent. So-called `automatons', or self-operating machines have existed for
a very long time. One of the first humanoid automatons ($\sim$800 C.E.) actually played the flute, and was programmable.\cite{automaton}
Other automatons have been created to write letters, 

Even today, the game of chess is sometimes used as a measure of intelligence.
In the 1700s, an automaton called ``The Turk'' was made, which appeared to play chess
against human players autonomously\cite{turk}



\section{Recent Technological Advances}
maybe talk about nvidia here

\section{Economic Factors Influencing the Industry}
https://www.newsmax.com/finance/richardagu/factors-artificial-intelligence-job/2018/04/24/id/856267/

\section{Grographic locations of industry concentration}
hey Edmonton actually has some pretty cool machine learning research stuff
going on
obviously there's Silicon Valley as well...

\section{Major empoloyers}
google, tesla...

\section{Projected Economic Status}
todo \cite{forbesprojected}

\section{Current hiring trends and long term prospects}
oof
\section{Conclusion}
\lipsum[1]


\singlespacing
\nocite{*}
\printbibliography
% \bibliographystyle{IEEEtran}
% \bibliography{references}
\end{document}
