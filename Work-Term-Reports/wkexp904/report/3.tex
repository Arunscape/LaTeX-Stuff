\section{Job Search Plan}

\subsection{Preparation}
An essential prerequisite for job hunting is a good resume. The plan here is to
keep it up to date, and keep track of accomplishments continuously. Even when I
find myself in a comfortable position, I should remind myself to update my
resume say, once every 6 months just to keep track of accomplishments. 

\subsection{Research and Networking}
Another important thing to keep and maintain is my network. As part of the job
searching plan, I would reach out to a few people in my network. I would let
them know I am currently open for opportunities. A good idea is to have my
resume reviewed by professionals who are willing to give advice. There are also
online tools available which make the research and networking process easier.

\subsection{Tools and Resources}
Speaking of online tools, a good tool to use is a social media sites like
Linkedin, which has a focus on professional development.  Platforms like these
make it easier to follow up with someone I connected with at a networking
event, and have the benefit of exposing the user (me) to new opportunities or
topics of interest. Glassdoor is also a good tool for researching about
potential positions, and for helping to decide if the opportunity is a good
fit.  It is also useful for checking if the salary and benefits I would be
earning are at market value.

\subsection{Potential Obstacles}
Things that usually go wrong when job hunting are: lack of offers, and the
opposite problem when there are multiple offers and an interview is coming up
soon before the deadline of another offer.  The first potential situation -- a
lack of offers can happen when the economy is performing poorly, or if not
enough effort is being put into job hunting. This can be solved by \todo{link
to a website which recommends how to job hunt during a recession.}

The second scenario is more tricky to deal with. Sometimes, it might not be a
bad idea to ask for more time to make a decision, or to tell one recruiter that
you have an offer lined up. However, that does not always work, and different
approaches need to be considered on a case-by-case basis. 
