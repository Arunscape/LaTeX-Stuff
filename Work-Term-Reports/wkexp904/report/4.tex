\section{Interviewing Graduates}
\subsection{Where did they begin their career?}
The trend here seems to be that initially, the starting points were not related
to the software engineering field, things like lawn mowing, sales,
construction, and consulting. However, these initial positions helped with
building the interviewees' networks. In this respect, it would seem that I was
lucky to have my first job be related to the field I am in. I have also made
excellent connections, and although my network is still admittedly small, it is
a great start.

\subsection{How long have they been working}
All interviewees started work either during a co-op term at university, or
after graduation. No one took any significant breaks, aside from leaving for a
few months to take care of their newborn children. This seems to be in line
with my plans as well. Two of them are in manager positions, while one is a
software developer. 

\subsection{Promotions attained and key factors contributing to them}
Promotions in most cases were attained by taking opportunities to try something
new, even if it was outside of the person's area of expertise and not in their
comfort zone. The appearance of new opportunities also comes with a little
luck, such as being in the right place at the right time when a senior leader
is retiring, or happening to be the only one with a vague idea of how to do
something because it was part of a hobby project. Sometimes, however, one
cannot advance their career by staying in the same spot. It seems that some
interviewees achieved more desirable positions by changing where they work when
they had the opportunity. My career plan seems to align well with the
experiences of the interviewees. Continuously learning, keeping hobbies, and
seeking out new opportunities every few years should set me up for success. 

\subsection{Type and nature of work duties/responsibilities over time}
The trend seems to be starting with minimal responsibilities in a junior
position. Over time, as the individual becomes more familiar with the
infrastructure, they are awarded a senior developer position. When the
opportunity presents itself, they took the chance to lead a project, team, and
eventually ended up in a manager role. 

\subsection{Are they where they thought they would be in their career? Why or Why not?}
All interviewees saw themselves working in an engineering related role, and
expected some form of career advancement over time. However, none of them began
in the software engineering field. It just happened to be an opportunity that
came up, and they found it interesting. It seemed like the positions they
are currently in were expected. 

\subsection{What sacrifices have they had to make?}
Sacrifices came in a few forms:
\begin{singlespace}
  \begin{itemize}
    \item spending less time with family
    \item working more hours than are being logged
    \item taking a pay cut to switch to a more interesting/fulfilling job
  \end{itemize}
\end{singlespace}
I believe no one plans to make any of the sacrifices listed above. 
However, I feel that I should take care to make sure I avoid the first two 
as much as possible. Mental health and personal relations are important. 
Taking a pay cut to switch for something more interesting, while not ideal
is more acceptable. 

\subsection{Where do they see themselves in 5, 10, 15 years?}
In 5 years, the interviewees see themselves in roughly the same position, maybe
with a promotion. In 10 years, a more senior position is expected, while in 15
years, no one has any idea what will happen.  However, everyone expressed the
desire to continue learning.  This echoes my original plan.

\subsection{Significant factors they see as important in determining success}
In general, the sentiment is that one is successful if they can look back and be happy with what they did. -- Working hard, doing the right thing, and being satisfied with their achievements.

\subsection{Have there been any challenges or delays in their career progression? If so, how did they overcome them?}
Generally, it seems that challenges come in the form of organizational changes
or lack of funding. These tend to result in changing of teams, projects, and
layoffs. Events like these cannot be prevented, however, one can be ready for
them by having an emergency fund. My original career plan mentioned that an
emergency fund would be a good idea.

\subsection{What would they do differently if they had to start again?}
The interviewees were content with their life choices.

\subsection{Key piece of advice for someone starting out in their career}
Recommendations included:
\begin{singlespace}
  \begin{itemize}
      \item making sure that the reputation I build for myself is done sustainably
      \item working long hours does not always put you ahead
      \item pick the brains of your leaders 
      \item stay curious
      \item a good, challenging, and interesting work environment is far more important than an increased salary
      \item finding a good fit is important, and if the job does not fit, it does not mean you're inadequate
      \item when offered a challenge, consider (but not always) saying yes. It's an opportunity to stretch and learn something new
      \item networking is important
      \item when networking, it is much more important to find people you enjoy talking to.
            Opportunities will come naturally. When trying to force a connection, things become awkward
  \end{itemize}
\end{singlespace}

\subsection{Surprises and Points I Have Not Considered Yet}
It surprised me me how consistent the responses between the interviewees were.
It also surprised me how close my original plan was to the general advice of
the people I interviewed. It surprised me that some were willing to work longer
hours, and not have it logged. Maybe I have not found an area which truly
captivates me yet, but also a work-life balance is important to have. Priorities
change over time, so this is just one of the many things to keep in mind.

\subsection{Unexpected Issues In Career Development Discovered}
Interviewing the engineers helped confirm I was on the right path with my
career plan, and also helped with confirming the risks I should be aware of.

\subsection{Changes to my career plans}
I should continue picking the brains of people around me. 
Stepping outside of my comfort zone more often to initiate conversations with knowledgeable people seems like the next logical step. This is something I can do with my teammembers, leaders, and the next person I meet at a networking event.
