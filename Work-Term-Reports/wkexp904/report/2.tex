\section{(Initial) Career Plan}

\subsection{Industry}
In the
\href{https://github.com/Arunscape/LaTeX-Stuff/blob/a732f319e8a6526f036bd424fa24f66941f09a5a/Work-Term-Reports/wkexp902/report.pdf}{wkexp902
report}, 
I expressed interest in the Artificial Intelligence, or AI industry, and at the
time of writing this report, it remains an area of interest for me. 


\subsection{Ultimate goal and how to get there} 
The ultimate career goal is obviously to get to a financially stable state,
living comfortably, and being able to afford a reasonable amount of luxuries.
There are many ways to get there, and several books claiming to share some
`secret', however, the general trend appears to be that these books tend to help
the authors reach their financial goals instead of the reader. A more pragmatic
approach should be taken which acknowledges the existence of uncertainty in the
real world. 

The first thing that should be acknowledged when making an initial career plan
is that I have little experience. So, the first step should be to get in touch
with experienced individuals, and ask for their advice. Luckily, that is the
purpose of this report, so the first step is already done.

The next thing to think about would be what I plan to do after graduation.
At this point, there are many areas in the software engineering field that I 
have not experienced yet. Even though I think that right now Artificial
Intelligence might be interesting, I might find out later that I find another
area to be more interesting. There is no way to figure that out without first
trying, so the plan for immediately after grduation is to try out working for
different companies in different areas in the software engineering field. 
The idea is to spend 1-2 years at various companies, and switch when it is
natural. For example, if I plan to move to a different city, or an exciting
opportunity arises. Doing this would also have the benefit of easily building
my network for people, and making that network as diverse as possible. In 
addition, this would have the benefit of allowing me to absorb as much knowledge
as possible, keep my skills up to date (as well as acquiring new ones), and
keeping me competitive in the job market. 

When I find myself comfortable, I think the plan is to never stop job searching.
This will allow me to stay competitive in the job market~\cite{kleiman_2015},
while I get to work in a field that interests me. It will also keep me on the
lookout for better opportunities.


\subsection{Roles to undertake}
For the first few years in my career, I plan to take hands-on roles which
involve programming in various languages. The next natural progression would be
to end up in various senior developer positions. After that, I could choose to
take on manager roles as well. By this point, I should be in a comfortable
position, financially so I could also choose to create a startup using the
experience gained from all the previous years of work, or use the same
experience to lead in a larger, already established company.

\subsection{Training}
The software engineering field is always rapidly changing. I would have to keep
myself up-to-date to keep myself competitive in the job market. One of my goals
is to work in an environment which encourages continuous learning. Ideally,
learning new skills would be encouraged at work, and the employer would provide
opportunities such as attending conferences for sharing knowledge and
networking. 

\subsection{Timelines}
I do not have any specific timelines in mind, however, it is a good idea to have
a general idea of where I would like to be in a few years. I think that I
should find myself in a senior developer position within 5 years from
graduation, however, there are so many variables to consider for a longer
timeline like 10 or 20 years. In that time, priorities, interests, (or anything
really) could change. I think around the 10 year mark I would at least like to
lead a large-scale project. At this time, I will not set a specific goal for the
20 year mark, because too many things can change in that time. One thing I think
I would like to keep constant however, is continuing to learn new things, and
keeping myself up-to-date in the field. 

\subsection{Interests, strengths, weaknesses}
My interests obviously include ``programming'', however as that is such a broad
topic, I should probably select a few subcategories that interest me.  At the
moment, some technology trends I have been following are artificial
intelligence, RISC-V, Rust, WebAssembly, cryptocurrency and more.  There is no
way to know which trends will appear in the future which will interest me, so
that is for me to figure out in the future. However, in general, I think an area
I might enjoy is server backends.  One of my strengths is a healthy amount of
curiosity and desire to learn how things work. However, this strength can be a
double-edged sword. One of my weaknesses is being tempted by a new language or
framework just because it is new, even when there is an existing solution with
more community support, and better documentation. 

\subsection{Potential setbacks}
Life happens, and there is no way to predict it. Situations that may arise
include recessions in the economy, being laid off, and supporting a family down
the line. If planned, starting a family is not a setback.  Risks like the
economy not doing great at certain times, and unexpected layoffs can be
mitigated by following generally agreed-upon financial best-practices like
maintaining 6 months worth of salary in an emergency fund that is liquid and
easily accessible, but also gains interest~\cite{r/personalfinancecanada}. This
would also help mitigate other unforeseen circumstances. 
