\documentclass[letterpaper,12pt]{article}
\synctex=1

\usepackage[margin=1in]{geometry}
\usepackage{lipsum}
\usepackage{setspace}
\doublespacing
\usepackage{parskip}
\parskip=\baselineskip

\usepackage[hyphens,spaces,obeyspaces]{url}
\usepackage{hyperref}

\usepackage{graphicx}


% noto sans font
% \usepackage[sfdefault]{noto}
% \usepackage[T1]{fontenc}

\bibliographystyle{IEEEtran}

\title{WKEXP 901 Work Term Report}
\author{Arun Woosaree \\ \\  XXXXXXX}


\begin{document}
\relax
\begin{titlepage}
 \maketitle
 \thispagestyle{empty} %because I wanted to use \maketitle
 \centering
 \large
 \vspace{1cm}
 Work term 1\\
 \vspace{1cm}
 Computer Engineering Software Co-op \\
 \vspace{1cm}
 Coordinator: Alyson Rumsby
\end{titlepage}

\section{Overview}

ATB Financial, or Alberta Treasury Branches is a financial institution in Alberta, Canada.
Established in 1938, ATB Financial is a crown corporation with 50.7 billion dollars in assets.\cite{annualreport2018}
ATB has over five thousand team members who support the one hundred and seventy-five branches, one hundred and forty-two
agencies, three Entrepreneur Centres, a Customer Care Centre, and mobile and online banking.
These services serve over seven hundred and forty thousand customers. \cite{annualreport2018, annualreport2017}

I joined ATB Financial through its ATB 101 program, with about one hundred other students.
ATB 101 is a program offered by ATB Financial for summer students of various disciplines. 
In addition to the work done by each individual
in our regular positions, there are events which introduce us to the company's culture.
Furthermore, we are put into teams of about ten individuals to solve actual business problems.
Having a side project keeps us on our toes, and pushes you to manage your time effectively.
The ``capstone project'' functions very similarly to a case competition, except that it runs 
for four months. As a result, the presentations are done at a much higher calibre, and we get the time to put in our best effort while working on our regular roles.

I am a developer working on the future ATB Online for Business Website. My team falls under the
transformation umbrella, which
makes up a large part of the ATB organization. Its purpose is to ``reimagine banking'' \cite{atbstory}.
Specifically, I am a part of the digital transformation team. 
Several teams work in digital transformation, and we work on various
things together, including mobile apps, websites, crucial systems for a bank to work, 
as well as working on entirely new things to make banking better. When one thinks of a bank,
programming is not usually in the list of things that come to mind. However, this could not be
further from the truth, since technology
plays a huge role in today's world of banking, and banking today simply would not be what it is
today without computer programmers.

%.    Provide an overview of the organization currently employing you. Include the following:
%•    Size and purpose of the organization
%•    Products and services provided and markets served
%•    Where in the organization your area/department fits, its role and responsibilities
%•    Your role in your department and in the organization
% K I think I covered all these


\section{Learning Opportunities}

In the short time I have had working so far for ATB Financial, the organization has presented
countless exciting challenges and learning opportunities.
From day one, I was put outside of my comfort zone, presented with the challenge of
getting accustomed to a new operating system. Having seldomly used macOS before,
there was a bit of a learning curve. Even though macOS is designed to be `` user-friendly'',
there are some nuances to get used to coming from a Linux or Windows environment.
As soon as compliance training was over, I was given the opportunity to take online
courses for almost two weeks to familiarize myself with the technologies that I am
working with now. I took refresher courses on JavaScript, and then I moved on to learning
about creating web applications using React and Redux. React is a JavaScript framework developed by Facebook,
which enables the creation of web user interfaces with reusable components\cite{react}, and Redux is a state container
for JavaScript apps\cite{redux} which works nicely with React. 
During my work term so far at ATB Financial, I have learned new languages and industry standard programming
practices, as well as intuitively organizing my code by making my git commits cleaner. Git is a piece
of open source software for version control, written by Linus Torvalds.\cite{git}
These valuable skills that I have learned and practiced are simply not taught in traditional classroom settings.


In addition to coding, I was introduced to the practice of agile workflow. The idea of agile is to work together
in teams, fail fast, and to regularly reflect on things as a team.\cite{agile}
In my team, we had daily scrum meetings, where we would update everyone with
what's going on, and if there are any blockers. Using an agile workflow allows us to react and adapt
quickly whenever there are changes. In the spirit of agile learning, the ATB 101
program allows us students to further improve our skills by working on a four-month ``capstone project''
As mentioned before, the ATB 101 program presents real business challenges that we learn to
solve in groups. My team's task was to reimagine rural branches. The goal was to figure out
how we can drive a better customer experience by incorporating arts and culture in rural branches,
as other banks abandon these communities \cite{saskbranches}. In our team of ten people,
we had meetings two times a week, as we worked together on coming up with a solution.
This experience was quite unique, since most other team members were business students and
I was the only computer programmer. In addition, to put our agile skills to the test,
there were sometimes some unexpected challenges, such as suddenly being told that we have to produce
a video in the last few weeks of the project.
At the end, each team presents the challenge they worked on and the solution they came up with.
As one may notice, this process was wildly different from my regular role as a programmer, which
provided a unique opportunity for me to improve on my skills outside of coding. Going into
this job as a programmer, I did not expect to be part of such an amazing team, in addition to
the team that I work with every day.


%    Describe the opportunities available for learning in the organization. Some of these may include:
%       Production/operation processes
%    Management style used
%    Methods used to coordinate operations/projects and budgeting/planning

\section{Work Preferences}

Being my first job, I did not know what to expect.
However, I was initially quite surprised with ATB Financial's structure. Everyone, including managers
and executives are very approachable and friendly. The Chief Executive Officer himself, works at
a desk just like everyone else who works in the office. The idea behind this is to leave the
corner spots with windows for the meeting rooms that everyone has access to.
Overall, I really like how ATB is structured. From what I gather, some aspects of the philosophy
behind ATB Financial's structure are outrageously different compared to some other corporate institutions.
This was exemplified during Culture Day.
As a part of the ATB 101 program, each student
must attend Culture Day, which happens once a year. Culture Day really highlighted the company's
core values, manifested as the eleven ``ATB's''\cite{11atb}. On this day, company executives and
other respected individuals shared multiple stories related to the 11 core values which we refer to
as the ``ATB's''. These stories put the company's values into perspective, which makes ATB's structure
make sense. Taken together, the eleven ATB's, Culture Day, and the ATB 101 program
highlight the fact that everyone at ATB is there to support each other.


The management style was very hands-off, at least in my role. As a student,
I thought that I was given a lot of responsibility, however, I am glad
with the level of trust I was given, as I found myself feeling
comfortable and motivated to work.
Initially, I was confused as to why no work hours were specified. As it turns out,
work hours are extremely flexible, and it is up to you to make sure that you are
putting in the hours, and to take breaks when you need them. You are given a laptop, so
that you are not tied to your desk. This freedom of flexibility allows you to be
as comfortable as possible, whether you prefer to work at home, in a coffee shop,
or in the office. Even though I was given the opportunity to work from home,
I found myself still showing up in person at the office
because I was able to have a human connection when attending meetings,
and asking for help. Even though I have not worked
anywhere else beforehand, I found myself being comfortable working in the
environment ATB Financial has offered. This does not surprise me, as it seems
that nearly every step is taken to make ATB Financial's employees feel
comfortable in the space they work in.


As a computer programmer, I have always expected to just be in the office, behind a computer,
converting coffee into code. However, I have realized that a job in my field
involves a lot more than that. I never thought that I would be able to work
on a user interface, but my job at ATB has proved me wrong.
Perhaps in the future, I will learn backend stuff, devOps, artificial intelligence,
machine learning, or something else. One thing that is for sure, though, is that
no matter what I do in the future the only constant I expect is to be
continually learning something new every day at work, as I have while working
at ATB Financial so far.


%\begin{enumerate}
% \item prefer management style similar to ATB where trust is given
% \item I like the challenges given by ATB, where every day I'm learning something new and I hope
%       to always continue learning
% \item I like the hands-off management style where you are responsible for showing up to work on time, keep track of your hours,
%       and take breaks when you should
% \item I'm in computer engineering, I don't see myself working in a plant or out in the field surveying anytime soon.
%\end{enumerate}

%       Based on this information, provide an analysis of your work preferences, which can help you in deciding the direction of your
%         future work terms. Discuss your preferences regarding:
%    Style of management/supervision
%    Type and nature of tasks
%    Level of responsibility
%       Work environment (field, plant, or office) and location
% k I think I'm done this
\singlespacing
\nocite{*}
\bibliography{references}
\end{document}
