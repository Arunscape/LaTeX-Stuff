\documentclass[letterpaper,12pt]{article}
\synctex=1

\usepackage[margin=1in]{geometry}
\usepackage{lipsum}

\usepackage{setspace}
\doublespacing
\usepackage{parskip}
\parskip=\baselineskip

\usepackage[hyphens,spaces,obeyspaces]{url}
\usepackage{hyperref}

\usepackage{graphicx}

\interfootnotelinepenalty=10000

\usepackage[
    style=ieee,
    backend=biber
    ]{biblatex}
\addbibresource{references.bib}

\title{WKEXP 903 Work Term Report}
\author{Arun Woosaree \\ \\ XXXXXXX}


\begin{document}
\relax
\begin{titlepage}
 \maketitle
 \thispagestyle{empty} %because I wanted to use \maketitle
 \centering
 \large
 \vspace{1cm}
 Work term 3\\
 \vspace{1cm}
 Computer Engineering Software Co-op \\
 \vspace{1cm}
 Coordinator: Linda Szekely
\end{titlepage}

\section{Introduction}
The purpose of this document is to serve as an introduction to the Association of Professional Engineers and Geoscientists Code of Ethics\cite{apegacode}.
From this point forth in the document, the Association of Professional Engineers and Geoscientists will be referred to by its acronym: APEGA.
In the following paragraphs, we will be analyzing a real ethical dilemma, and relating it to APEGA's Code of Ethics. This should give a good idea
of the importance of ethics in the engineering profession. Part of the training for being an engineer is learning how to use the APEGA Code of Ethics
to aid the decision-making process for ethical dilemmas. As the talking points discussed may be sensitive, the names of relevant parties and employers involved have been modified to ensure that confidentiality is maintained.
Fun fact: Maintaining confidentiality is part of Rule 3 in Section 4.3.3 in the APEGA Code of Ethics, and will be further explored in this report.
There are five main rules one must consider in the APEGA Code of Ethics:
\begin{enumerate}
    \setlength{\itemsep}{0pt}
    \setlength{\parskip}{0pt}
    \setlength{\parsep}{0pt} 
    \item Health, Safety \& Welfare of the Public
    \item Competence and Knowledge
    \item Integrity, Honesty, Fairness, and Objectivity
    \item Statutes. Regulations, Bylaws
    \item Honour, Dignity, and Reputation
\end{enumerate}

\section{The Situation}
A professional engineer whom we will be referring to as \textit{Kowalski} is currently working for ABC Company.
Recently, Kowalski received a job offer from XYZ Industries, which is ABC Company's main competitor. As it turns out,
the offer is a good one, and he would like to accept the offer. Initially, Kowalski thought he could not take 
the job on, because he was not sure if he signed an agreement with ABC Company that limited his ability to work for the
competitor for a certain amount of time. Luckily for Kowalski, no such agreement was made, so he is free to work for XYZ Industries.
However, in his new position, Kowalski finds himself challenged with a few ethical dilemmas where there is no clear solution.
Fortunately for Kowalski, the APEGA Code of Ethics can be used to aid with his decision-making process, since he is
a good engineer who has internalized the APEGA Code of Ethics and knows how to use it to help make tough decisions.


\section{Analysis}
\subsection{Health, Safety \& Welfare of the Public}
Regarding this rule of the APEGA code of ethics, nothing really should change in Kowalski's new position at XYZ Industries.
Assuming that Kowalski has been a good engineer, and has always held ``paramount the health, safety and welfare of the public, and have regard for the
environment"\cite{apegacode}. As long as he continues to uphold this rule, there should not be anything new to consider here. That is,
also assuming that XYZ Industries has no issues in this area. However, if there are any issues in this regard at XYZ Industries, Kowalski is obligated to take appropriate action.
This would involve letting his employer or client know about the risk to the public interest.
If any issues arise that hinder Kowalski's ability to do the right thing, such as a disagreement between the ``degree of risk presented by a project versus the degree of
protection of the public interest which is warranted"\cite{apegacode}, it should be noted that Code of Ethics has a framework for ``respectful and constructive disagreement", 
and a process for resolution titled ``Having Recommendations Overruled" in section 4.3.5.

\subsection{Competence and Knowledge}
Before accepting his new role at XYZ Industries, Kowalski should make sure that he would be competent in his new role.
Of course, this will likely not be a concern if the jobs are very similar. However, if the new role has additional responsibilities that Kowalski
is not trained for, Kowalski should inform his employer or client. This rule does not prevent Kowalski from taking on his new role. From an ethical standpoint,
Kowalski should not claim to be qualified for something that he is not competent in doing. However, it is totally fine for him to learn new skills and take on
new tasks and challenges as long as he makes it clear that some learning would be involved, and his employer or client knows about it. 
In this case, an expert can be "engaged" to help out with the situation, as per section 4.2.5 of the APEGA Code of Ethics.\cite{apegacode}

\subsection{Integrity, Honesty, Fairness and Objectivity}
Regarding the points made in this section of the APEGA Code of Ethics, the one that Kowalski may find quite tricky to
adhere to is \textit{maintaining confidentiality}. In his new role, Kowalski needs to be careful to not share
any confidential or proprietary information about ABC company with his new employer. This includes any designs such as blueprints,
process information, or any information directly obtained from his previous employer or clients. 
This information is considered the property of his previous employers or clients and was loaned to him for the purpose of working on the project.
Although, if Kowalski obtains permission from ABC Company to share confidential information with XYZ Industries, he would be allowed to share the information that ABC Company agreed to. 
However, this is unlikely to occur, since ABC Company and XYZ Industries are close competitors. The only exception to this rule is that Kowalski is allowed to share confidential information if the law requires him to (to the extent that the law requires), or if something is posing a risk to the public interest. It should be noted that this
rule does not prevent Kowalski from sharing or using technical knowledge obtained during his time at ABC Company. Technical knowledge is considered a part of
Kowalski's own professional experience, and he does not need permission to use his technical knowledge. It is absolutely unacceptable, however, for Kowalski to make a copy of
software or code from his previous job to use later. Again, this does not prevent him from using the technical knowledge he acquired to write new software, 
but it does restrict him from either using a copy of the software from his previous jobs, and also from using a modified version of the software
from those jobs. In that case, it would not only be a breach of the code, but also a breach of the Canadian Copyright Act\cite{copyrightact}.
Even if Kowalski kept software or code from ABC Company or previous jobs for himself, he could be tried for ``unjust enrichment".\cite{apegacode}

\subsection{Statutes, Regulations, Bylaws}
Similar to the first rule, it is likely that there is nothing new for Kowalski, as long as he was, and continues to be a good engineer.
As mentioned in the previous paragraph, it is not only dishonest for Kowalski to disseminate secrets, or proprietary information unless required
by the law or for public interest's sake, but there are also laws such as the Canadian Copyright Act and others which make doing so illegal. It is
Kowalski's responsibility to be aware of, and to stay up-to-date with any laws that may have an effect on the new work that Kowalski will be taking on.
If Kowalski is taking on a new project, and learning a new task (which he made clear is outside his current area of expertise), it is also his duty to
familiarise himself with any laws that may apply to the new project.

\subsection{Honour, Dignity and Reputation}
Once again, as long as Kowalski was, and still is a good engineer, there should not be much that is new for him to consider regarding this rule
in his new position. He should continue to ``uphold and enhance the honour, dignity,
and reputation of their professions, and thus the ability of the professions to serve the
public interest''\cite{apegacode}. This includes acting like a professional should, whether at work or not.
This also means that an engineer should have their personal views and professional activities separated and that an
engineer's professional opinions should be factual to the best of their ability based on their knowledge and past experiences,
as mentioned in section 4.3.2 of the APEGA Code of Ethics\cite{apegacode}. This does not prevent an engineer from expressing their
personal opinions, however, all professional work should be devoid of bias due to political, economic or other factors which are
non-technical.

\section{Conclusion}
As we can see, while the APEGA Code of Ethics is clearly defined, there are situations which arise that must be considered carefully by an engineer or
geoscientist when making ethical decisions. In this report, we took a deep dive into a real situation encountered by a professional engineer.
We analyzed each of the rules in the APEGA Code of Ethics and looked at how different parts of the Code can apply to the situation.
Hopefully, at this point, we now have a better idea of how to apply the APEGA Code of Ethics and use it to help with making
ethical decisions in the future.


\singlespacing
\nocite{*}
\printbibliography
\end{document}
