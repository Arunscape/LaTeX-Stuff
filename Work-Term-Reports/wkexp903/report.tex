\documentclass[letterpaper,12pt]{article}
\synctex=1

\usepackage[margin=1in]{geometry}
\usepackage{lipsum}

\usepackage{setspace}
\doublespacing
\usepackage{parskip}
\parskip=\baselineskip

\usepackage[hyphens,spaces,obeyspaces]{url}
\usepackage{hyperref}

\usepackage{graphicx}

\interfootnotelinepenalty=10000

\usepackage[
    style=ieee,
    backend=biber
    ]{biblatex}
\addbibresource{references.bib}

\title{WKEXP 903 Work Term Report}
\author{Arun Woosaree \\ \\ XXXXXXX}


\begin{document}
\relax
\begin{titlepage}
 \maketitle
 \thispagestyle{empty} %because I wanted to use \maketitle
 \centering
 \large
 \vspace{1cm}
 Work term 3\\
 \vspace{1cm}
 Computer Engineering Software Co-op \\
 \vspace{1cm}
 Coordinator: Linda Szekely
\end{titlepage}

\section{Introduction}
The purpose of this document is to serve as an introduction to the Association of Professional Engineers and Geoscientists Code of Ethics\cite{apegacode}.
From this point forth in the document, the Association of Proffestional Engineers and Geoscientists Code of Ethics will be referred to by its acronym: APEGA.
In the following paragraphs, we will be analyzing a real ethical dilemma, and relating it to APEGA's Code of Ethics.
As the talking points discussed may be sensitive, the names of relevant parties and employers involved have been modified to ensure that confidentiality is maintained.
Fun fact: Maintining confidentiality is part of Rule 3 in Section 4.3.3 in the AEGA Code of Ethics, and will be further explored in this report.
There are 5 main rules one must consider in the APEGA Code of Ethics:
\begin{enumerate}
    \setlength{\itemsep}{0pt}
    \setlength{\parskip}{0pt}
    \setlength{\parsep}{0pt} 
    \item Health, Safety \& Welfare of the Public
    \item Competence and Knowledge
    \item Integrity, Honesty, Fairness, and Objectivity
    \item Statutes. Regulations, Bylaws
    \item Honour, Dignity, and Reputation
\end{enumerate}

\section{The Situation}
A professional engineer whom we will be referring to as \textit{Kowalski} is currently working for ABC Company.
Recently, Kowalski received a job offer from XYZ Company, which is ABC Company's main competitor. As it turns out,
the offer is a good one, and he would like to accept the offer. Initially, Kowalski thought he could not take 
the job on, because he was not sure if he signed an agreement with ABC Company that limited his ability to work for the
competitor for a period of time. Luckily for Kowalski, no such agreement was made, so he is free to work for XYZ Company.
However, in his new position, Kowalski finds himself challenged with a few ethical dilemmas where there is no clear solution.
Fortunately for Kowalski, the APEGA Code of Ethics can be used to aid with his decision making process. 


\section{Analysis}

\subsection{Health, Safety \& Welfare of the Public}
Regarding this rule of the APEGA code of ethics, nothing really should change in Kowalski's new position at XYZ Company.
Assuming that Kowalski has been a good engineer, and has always held ``paramount the health, safety and welfare of the public, and have regard for the
environment"\cite{apegacode}, as long as he continues to uphold this rule, there should not be anything new to consider here. That is,
also assuming that XYZ Company has no issues in this area. However, if there are any issues in this regard at XYZ Company, Kowalski is obligated to take appropriate action.
If any issues arise that hinder Kowalski's ability to do the right thing, it should be noted that Code of Ethics has a framework for respectful and constructive disagreement, and a process for
resolution titled ``Having Recommendations Overruled" in section 4.3.5

\subsection{Competence and Knowledge}
Before accepting his new role at XYZ Company, Kowalski should make sure that he would be competent in his new role.
Of course, this will likely not be a concern if the jobs are very similar. However, if the new role has additional responsibilities that Kowalski
is not trained for, Kowalski should inform his employer. This rule does not prevent Kowalski from taking on his new role. From an ethical standpoint,
Kowalski should not claim to be qualified for something that he is not competent in doing. However, it is totally fine for him to learn new skills and take on
new tasks and challenges as long as he makes it clear that some learning would be involved, and his employer knows about it. In this case, an expert can be "engaged"
to help out with the situation, as per section 4.2.5 of the APEGA code of ethics.\cite{apegacode}

\subsection{Integrity, Honesty, Fairness and Objectivity}
Regarding the points made in this section of the APEGA code of ethics, the one that Kowalski may find quite tricky to
adhere to is \textit{maintaining confidentiality}. In his new role, Kowalski needs to be careful to not share
any confidantial or proprietary information about ABC company with his new employer. This includes any designs such as blueprints,
any process information, or any information directly obtained from his previous employer. However, if Kowalski obtains permission from ABC Company to share
confidantial information with XYZ Company, he is allowed to share the information that ABC Company agreed to. This unlikely to occur though,
since ABC Company and XYZ Company are close competitors. The only exception to this rule is that Kowalski is allowed to share confidantial information 
if the law requires him to (to the extent that the law requires), or if something is adversely affecting public interest. It should be noted that this
rule does not prevent Kowalski from sharing or using technical Knowledge obtained during his time at ABC Company. technical knowlege is considered as part of
Kowalski's own professional experience, and he does not need permission to use it. It is absolutely unnaceptable, however for Kowalski to make a copy of
software code from his previous job to use later. Again, this does not prevent him from using the technical knowlege he aqcuired to write new software, 
but it does restrict him from either using a copy of the software from his previous jobs, and also from using a modified version of the software
from his previous jobs. In that case, it would not only be a breach of the code, but also a breach of the Canadian Copyright Act\cite{copyrightact}.
Even if Kowalski kept software or code from ABC Company or previous jobs for himself, he could be tried for ``unjust enrichment".\cite{apegacode}

\subsection{Statutes, Regulations, Bylaws}
Similar to the first rule, it it likely that there is nothing new for Kowalski, as long as he was, and continues to be a good engineer.
As mentioned in the previous paragraph, it is not only dishonest for Kowalski to disemmate secrets, or proprietary information unless required
by law or for public interest's sake, but there are also laws such as the Canadian Copyright Act and others which make this illegal. It is
Kowalski's responsibility to be aware of, and to stay up-to-date with any laws that may have an effect on the new work that Kowalski will be taking on.
If Kowalski is taking on a new project, and learning a new task (which he made clear is outside his current area of expertise), it is also his duty to
familiarise himself with any laws that apply to the new project.

\subsection{Honour, Dignity and Reputation}
kms

\section{Conclusion}
As we can see, while the Code of Ethics is clearly defined, there are situations which arise that must be considered carefully by an engineer or
geoscientist when making ethical decisions


\singlespacing
\nocite{*}
\printbibliography
\end{document}
